\documentclass[11pt, a4paper]{article}
\usepackage[main=galician, spanish]{babel}
\usepackage{fontspec}
\usepackage{libertinus}%%automatically loads aunicode-math package
%%\renewcommand{\familydefault}{\sfdefault}
\usepackage{float}
\usepackage{parskip}
\usepackage[version=4]{mhchem}
\usepackage[figuresleft]{rotating}
\usepackage{graphicx}
\usepackage{amsmath}
\usepackage{slashed}
%\usepackage{amsfonts}
%\numberwithin{equation}{section}%%numbering eqs by section
\usepackage{fancyhdr} 
\usepackage[dvipsnames, table]{xcolor}
\usepackage{wrapfig}
\usepackage[labelfont=bf, skip=4pt, figureposition=bottom, font=normalsize]{caption}
\usepackage{siunitx}
\usepackage{adjustbox}
\usepackage{booktabs} 
\usepackage{multirow}
\usepackage[a4paper,left=2cm,right=2cm,top=2.5cm,bottom=2.5cm]{geometry}
\usepackage{csquotes}
\usepackage{enumitem}
\usepackage{subcaption}
\usepackage{listings}
\usepackage{abstract}
\usepackage{hyperref} 
%%adapted to galician (GZ)
\renewcommand{\tableautorefname}{Cadro}
\usepackage{lipsum}

%% ENUMITEM settings
%\setlist{itemsep=4pt}
\setlist[enumerate,1]{
  label={\bfseries \arabic*.},
}

%% SIUNITX
\sisetup{output-decimal-marker = {,}}
\sisetup{exponent-product= \cdot}
\sisetup{separate-uncertainty=false}
\sisetup{table-parse-only=true}
\sisetup{inter-unit-product = \ensuremath { { } \cdot { } } }
\sisetup{detect-all}
\sisetup{group-digits=integer}
\sisetup{print-unity-mantissa=false}
\sisetup{list-final-separator = { e }}
\sisetup{range-phrase = -}
\sisetup{range-units = single}
\sisetup{list-units = single}
\sisetup{propagate-math-font = true}
\sisetup{text-series-to-math = true}
\sisetup{list-final-separator = { e }}
\sisetup{list-pair-separator = { e }}
\DeclareSIUnit{\espesor}{\milli\g\per\cm\squared}
\DeclareSIUnit{\aten}{\cm\squared\per\milli\g}
\DeclareSIUnit\clight{\text{\ensuremath{c}}}
\DeclareSIUnit\year{a}
\DeclareSIUnit{\barn}{b}
\DeclareSIUnit{\bar}{bar}
\DeclareSIUnit{\neutrons}{\text{neutróns}}
\DeclareSIUnit{\deuterons}{\text{deuteróns}}
\DeclareSIUnit{\counts}{\text{contas}}

%%OTHER COMMANDS
%% for isotopes
\newcommand{\iso}[2]{\ce{^{#1}#2}}
%for vectors
\newcommand{\vect}[1]{\boldsymbol{#1}}
%for ann
\newcommand{\ann}{a_{nn}}
% figure sizes
\newcommand*{\fw}{0.6}
%%for lstlisting package
\definecolor{backcolour}{rgb}{0.95,0.95,0.92}
\lstset{
%backgroundcolor=\color{backcolour},
basicstyle=\small\ttfamily,
language=C++,
keepspaces=false,
keywordstyle=\bfseries\color{Purple},
morekeywords={TSpectrum, TH1, TH1I, TH1D,
Double_t, Int_t, TGraph, TGraphErrors, TMinuit, TF1, TSpline3, G4NistManager,
QGSP_BIC_HP,
SimGeometry},
stringstyle=\color{Orange},
identifierstyle=\color{ForestGreen},
breaklines=false,
captionpos=b,
belowskip=-0.85 \baselineskip,
%columns=flexible,
}

\title{\textbf{Guión das prácticas no IGFAE}}
\author{%Miguel Lozano González\\
\small{Prácticas optativas de Grao $\cdot$ Curso 23-24}\\
\small{Universidade de Santiago de Compostela}}
\date{\empty}%\date{\today}

\pagestyle{fancy}
\fancyhf{}
%\fancyhead[R]{\thepage}
\fancyhead[R]{\textsc{guion prácticas igfae}}
\fancyfoot[C]{\thepage}
\renewcommand{\headrulewidth}{0.75pt}
%\renewcommand{\footrulewidth}{0.75pt}
\setlength{\headheight}{14.5pt}

\hypersetup{
	colorlinks=true,
    linkcolor=blue,
    filecolor=magenta,      
    urlcolor=cyan,
}

%% BIBLIOGRAPHY
% \usepackage[
% backend=biber,
% style=numeric,
% sorting=none
% ]{biblatex}
% \addbibresource{./biblio.bib}

%% DOCUMENT

\begin{document}
\begin{minipage}{0.48\linewidth}
    \maketitle
\end{minipage}\hfill
\begin{minipage}{0.48\linewidth}
    \tableofcontents
\end{minipage}

\noindent\rule{\textwidth}{1pt}
% {\renewcommand{\abstractname}{}
% \renewcommand{\absnamepos}{empty}
% \begin{abstract}
% \noindent Memoria das prácticas externas do Mestrado en Física realizadas no Grupo de Física Corpuscular e Aplicacións (FICA), adscrito ao IGFAE (\url{https://igfae.usc.es/igfae/}), baixo a supervisión de Juan Lois Fuentes (\url{juan.lois.fuentes@usc.es}), sendo titora académica Beatriz Fernández Domínguez (\url{beatriz.fernandez.dominguez@usc.es}).
% \end{abstract}
% }
\section{Presentación}
\subsection{Obxectivos}
O obxectivo destas prácticas vai ser reproducir a resposta real do detector \textit{ACtive TARget and Time Projection Chamber} (ACTAR TPC) nun computador de cara a preparar a campaña experimental de 2025. Simularase unha reacción nuclear de cuxos resultados avaliaremos a resolución na reconstrucción de ángulos.

Abarcaremos varias áreas:
\begin{itemize}
    \item \textbf{Programación}: Aprenderemos a programar en C++ con ROOT. Utilizaremos progamas sinxelos para que a adaptación sexa o máis liviana posíbel e fomentaremos o recurso á documentación na web.
    \item \textbf{Detectores}: ACTAR TPC é un detector gasoso moi recente que abre todo unha nova ventá de posibilidades no eido das reaccións nucleares. Introduciremos os conceptos básicos dun detector gasoso e complementarémolos cos de detectores de Silicio.
    \item \textbf{Teoría}: Expoñeremos as leis básicas da cinemática que rixen calquera reacción entre partículas e introduciremos os fundamentos da interacción radiación-materia, básicos para a detección experimental.
\end{itemize}

Iremos adaptando o guión das prácticas en función das necesidades e do tempo dispoñíbel. Gran parte do código computacional está xa listo, e o que se requerirá será facer uso del en programas máis simples.

\subsection{Motivación}
En 2025 levarase a cabo un experimento coa reacción $\iso{11}{Li}(d, t)\iso{10}{Li}$ para estudar a estrutura nuclear do \iso{10}{Li}. En particular, este tipo de reaccións de transferencia dun nucleón permiten acceder de forma pormenorizada á información espectroscópica do núcleo resultante; neste caso, co obxectivo principal de estudar o estado fundamental do \iso{10}{Li}.

Para lograr este obxectivo, necesitamos varias compoñentes:
\begin{itemize}
    \item \textbf{Detector}: O gas que encherá ACTAR TPC será unha mestura de \ce{D_2} e \ce{iC4H10} (\qty{95}{\percent} e \qty{5}{\percent}, respectivamente) a unha presión de \qty{900}{\milli\bar}. Será o \ce{D_2} o que provea os $d$ da reacción.
    \item \textbf{Reacción}: A xa mencionada:
          \begin{equation*}
              \iso{11}{Li} + \iso{2}{H} \longrightarrow \iso{3}{H} + \iso{10}{Li}
          \end{equation*}
    \item \textbf{Feixe}: O \iso{11}{Li} terá unha enerxía de \qty{7.5}{A\MeV} (é dicir, \qty{82.5}{\MeV}).
\end{itemize}

\section{Principios físicos}
\subsection{Un detector gasoso}
ACTAR TPC segue os principios de funcionamento dunha \textit{time projection chamber}: as partículas ao pasaren polo gas ionizan os átomos, liberando electróns que \textbf{derivan} baixo aplicación dun campo eléctrico. Esta carga é recollida nun sensor amplamente fragmentado (), permitindo \textit{seguir} as partículas no seu traspaso dentro do medio en 2D.

Pero é máis: pódese engadir a terceira coordenada coñecendo o tempo que lle leva aos electróns chegar ao \textit{pad plane}, pois a \textbf{velocidade de deriva} é constante. Deste xeito, o seguimento das partículas faise en \textbf{3D}.

Finalmente, o principio de \textit{active target} engade a vantaxe de ser o gas o propio albo da reacción: este actúa como medio de reacción e de detección. O seguinte esquema ilustra este modo de funcionamento.
\begin{figure}[!ht]
\begin{minipage}[b]{.45\textwidth}
\centering
\includegraphics[width=1\textwidth]{figures/tpc.png}
\caption{Esquema de funcionamento dunha TPC.}
\label{fig:tpc}
\end{minipage}
\hfill
\begin{minipage}[b]{.45\textwidth}
\centering
\includegraphics[width=1\textwidth]{figures/actar.png}
\caption{Deseño do detector ACTAR TPC. A rexión de detección é a parte negra interna.}
\label{fig:actar}
\end{minipage}
\end{figure}

Para a simulación vamos necesitar poñerlle números ao detector:
\begin{itemize}
    \item O tamaño da zona de detección (\autoref{fig:actar}) é de \qtyproduct[product-units=power]{256 x 256 x 255}{\mm}.
    \item O criterio de dimensións é que o beam vai no eixo X, e a dimensión vertical é Z.
    \item Os detectores auxiliares poden poñerse arredor desta zona nos diferentes lados. Nós usaremos algo parecido ao da figura, cun detector de Si (como os \textcolor{blue}{azuis}) cara adiante.
    \item Situaremos un só detector de Si, cun tamaño de \qtyproduct[product-units=power]{80 x 50 x 1.5}{\mm} a unha distancia de \qty{10}{\cm} do \textit{pad plane}.
\end{itemize}

\subsection{Perdas de enerxía na materia}
As partículas ao propagarse a través do gas interaccionarán co campo de Coulomb dos átomos deste, perdendo enerxía e liberando electróns. Esta perda de enerxía en cada interacción é moi pequena, pero como é moi elevado o número de colisións no gas (é, polo tanto, un proceso estocástico), chega a ser moi importante. Nunha aproximación continua, está dada pola \textbf{ecuación de Bethe-Bloch}:
\begin{equation*}\label{eq:stopping}
    - \frac{dE}{dx} \cong \frac{4 \pi e^4}{m_e} \left(\rho \frac{N_A}{M}\right)\frac{z^2 Z}{v^2} \ln{\frac{2m_e v^2}{I}}
\end{equation*}
Depende esencialmente da partícula a considerar ($Z$ e $A$) e do gas (a través de $\rho$). É interesante definir:
\begin{itemize}
    \item Rango ($R$): É o camiño percorrido por unha partícula ata que \textit{case} se detén. É función da enerxía inicial e do material a atravesar. Existe unha \textbf{relación unívoca} $E \Longleftrightarrow R$ que imos utilizar para estimar a perda da enerxía en función da distancia percorrida pola partícula.
    \item \textit{Straggling}: Como a perda de enerxía é un \textbf{proceso estatístico}, dúas partículas coa mesma enerxía inicial non van depositar a mesma $\Delta E$. O \textit{straggling} mide, dalgunha forma, a $\sigma$ desta distribución.
\end{itemize}
Esta información está tabulada nun programa que se coñece como \textit{\textbf{SRIM}} e que utilizaremos sistematicamente nas prácticas. Basicamente, contén táboas cos parámetros que vamos necesitar para os cálculos: enerxías, rangos e stragglings.

Unha pequena apreciación: SRIM vainos dar o \textit{straggling} en posición, mais nós deberémolo converter a en enerxía. Aquí o algoritmo que imos usar:
\begin{enumerate}
    \item Calculamos $R_{ini}$ coa enerxía inicial, xunto co seu \textit{straggling}.
    \item Sabendo que a partícula percorre unha distancia $d$ o rango nese punto será $R_{L} = R_{Ini} - d$. Avaliamos o \textit{stragg.} para este valor.
    \item O \textit{straggling} na distancia estará \textbf{correlacionado con ambos}, e sabendo que $u^2(R_{Ini}) = u^2(R_{Left}) + u^2(d)$, despexamos $u(d)$.
    \item Aleatorizamos o valor de $d$ cunha gaussiana centrada no seu valor e de $\sigma = u(d)$. Recalculamos o valor de $R_{Left} \longrightarrow R_{Left}^{\prime}$ coa nova $d \longrightarrow d\prime$.
    \item Con este novo rango calculamos a enerxía final, efectivamente implementando o \textit{straggling}!
\end{enumerate}

\subsection{Resolución en enerxía}
Complementariamente acostuman situarse detectores de Si que nos van permitir medir a enerxía das partículas á súa saída do \textit{pad plane} (seguindo os mesmos principios que na anterior Sección). Un parámetro importante destes é a \textbf{resolución en enerxía}, definida como a capacidade para \textit{resolver} enerxías depositadas distintas: $R = \textrm{FWHM} / E$.

Para o noso experimento, esta resolución está tabulada en \qty{50}{\keV} a \qty{5.5}{\MeV} e podemos extrapolala ao resto de enerxías coa función:
\begin{equation*}
    R = \frac{2.35 \sigma}{E} = \frac{K}{\sqrt{E}} \quad \implies \quad \sigma = \frac{K}{2.35}\sqrt{E} = \frac{\qty{0.0213}{\MeV}}{2.35} \sqrt{E}
\end{equation*}
Isto vai significar que a perda de enerxía $\Delta E$ que calcules nos Si deberala aleatorizar cunha gaussiana de $\sigma$ obtida coa anterior fórmula.

\subsection{Cinemática}
Todo proceso de interacción entre partículas podes escribirse en termos de variables cinemáticas (enerxías e ángulos) usando as \textbf{leis de conservación} de enerxía e de momento. Esta pode ser descrita no sistema LAB ou no CM (mediante unha transformación Lorentz a un sistema con 3-momento nulo en ambas canles de entrada e saída) do seguinte xeito:

\begin{equation*}
    \begin{gathered}
        \text{No LAB:  }p_1=\left(E_1, \vect{p_1}\right); \; p_2=\left(m_2, \vect{0}\right); \; p_3=\left(E_3, \vect{p_3}\right); \; p_4=\left(E_4, \vect{p_4}\right) \\
        \text{No CM:  }p'_1=\left(E'_1, \vect{p'_1}\right); \;  p'_2=\left(E'_2, -\vect{p'_1}\right); \; p'_3=\left(E'_3, \vect{p'_3}\right); \; p'_4=\left(E'_4, -\vect{p'_3}\right)
    \end{gathered}
\end{equation*}

É estándar asumir unha partícula \textit{target} (a 2) en repouso. O que vamos necesitar para a simulación é:
\begin{itemize}
    \item Partiremos da enerxía cinética $T_1$ do feixe (1) no LAB.
    \item Calculares a transformación Lorentz da canle de entrada ao CM.
    \item Repartiremos a $E_{CM}$ (enerxía total no centro de masas) entre as partículas de saída cun $\theta_{CM}$ e $\phi_{CM}$ aleatorios.
    \item Finalmente, recuperaremos as enerxías de saída de ambas partículas no LAB coa transformación inversa.
    \item Tamén necesitaremos os ángulos no laboratorio, $\theta_{Lab}$ e $\phi_{Lab} = \phi_{CM}$.
\end{itemize}

Algunhas fórmulas que che poden ser interesantes para unha transformación Lorentz en X (o feixe móvese nesa dirección; é un convenio):
\begin{gather*}
    E^{\prime} = \gamma \left(E - \beta p_x\right), \qquad p^{\prime}_{x} = \gamma \left(p_x - \beta E\right)\\
    E = \gamma \left(E^{\prime} + \beta p_{x}^{\prime}\right), \qquad p_x = \gamma \left(p^{\prime}_x + \beta E^{\prime}\right)
\end{gather*}
Onde as variables primadas indican que son no CM. Cando transformes ao final do CM ao LAB terás en conta $\theta_{CM}$: $p_3 = p^{\prime}_3 \cdot\cos(\theta_{CM})$. O ángulo $\phi_{CM}$ non aparece de momento posto que é transversal á transformación.

\section{Para comezar}
\subsection{Introdución a ROOT}
O primeiro de todo é aprender a programar en C++ con \href{https://root.cern.ch/}{ROOT}. No \href{https://github.com/Practicas24/Practicas_24}{repositorio de Github}, dentro da carpeta \textit{Macros}, tes algúns exemplos de como traballar con el. Comeza polo básico e despois vas ver como podes crear histogramas, gráficos e números aleatorios: o básico que necesitamos para estar prácticas.

Velaquí algunha información básica que necesitas para comezar:
\begin{itemize}
    \item ROOT é un código desenvolto polo CERN que permite executar código de C++ sen compilar, mediante o que se coñece como \textbf{macros}:
          \begin{lstlisting}
        $ root -l NomeDoMacro.cxx
    \end{lstlisting}

          Vai executar a función chamada \textit{NomeDoMacro} dentro dese ficheiro. Polo tanto, debes nomear o ficheiro igual que a función principal que conteña dentro. Isto non impide que definas outras funcións dentro do mesmo.
    \item Debes saír sempre da sesión unha vez o programa termina, escribindo
          \lstinline|.q|
    \item ROOT provee de numerosas \textbf{clases} que realizan múltiples funcións: \lstinline|TH1D| (histogramas), \lstinline|TGraph| (gráficos), ... Sempre podes consular a documentación na web cando non saibas como se constrúe ou que funcións ten unha clase. Escribe no teu buscador favorito: \textit{NomeDaClase root cern} e deberías ter a documentación no primeiro resultado.
\end{itemize}

\subsection{Github}
Git é unha ferramenta de control de versións que che permitirá manter un historial do teu código, podendo volver atrás cando sexa necesario. Github é unha web para almacenar repositorios Git. Vamos utilizala para poder revisar o código a distancia.

Para iso, o primeiro é ter unha conta en \url{www.github.com}. Tras unha configuración inicial un pouco tediosa, o plan de traballo é o seguinte, que se debe executar cando fagas cambios importantes ou cando remates a túa sesión de traballo:
\begin{enumerate}
    \item \lstinline|git add .| vai engadir todos os cambios dende o anterior \textit{commit}
    \item \lstinline|git commit| abrirá un editor de texto no que crear unha mensaxe para informar dos cambios. Péchase con \lstinline|Ctrl+S, Ctrl+X|
    \item \lstinline|git push| envía os cambios á nube
\end{enumerate}

\section{Programa}
Desenvolveremos a simulación de forma modular, separando as distintas partes en diferentes macros e xuntando todo ao final.

A proposta é a seguinte:
\begin{itemize}
    \item \textbf{Primeira semana}: Configuración do entorno e familiarización con ROOT. Macros para constuír gráficos e samplear en histogramas.
    \item \textbf{Segunda semana}: Resolución da cinemática e implementación da xeometría.
    \item \textbf{Terceira semana}: Introdución das perdas de enerxía e das distintas resolucións.
    \item \textbf{Cuarta semana}: Estudo sistemático da resolución en $\theta_{Lab}$ para distintas configuracións da xeometría, gases, etc.
    \item \textbf{Quinta semana}: Posibilidade de engardir máis cousas á simulación (en función da evolución) ou comezo da memoria.
\end{itemize}

\subsection{Primeira semana}
Os obxectivos son os seguintes:

\subsubsection*{Cinemática}
Resolverás as ecuacións de cinemática para obter as enerxías cinéticas e ángulos no LAB das dúas partículas finais, aínda que presentando máis atención á pesada. Faino de forma xeral, tal e como está escrito na Sección correspondente. O criterio de números é o seguinte: $1 + 2 \longrightarrow 3 + 4$.

Na simulación, as variables que teremos que introducir nas fórmulas serán as masas, a enerxía do feixe e o ángulo $\theta_{CM}$ (que será aleatorio, como veremos).

\subsubsection*{Macros iniciais}
Quero que fagas dous macros onde me representes as seguintes cousas, usando a axuda do repositorio de Github:
\begin{enumerate}
    \item Representacións \textbf{cinemáticas}: Creas un \lstinline|TCanvas| con 3 \textit{pads} e representas a cinemática para estas tres reaccións distintas: $\iso{11}{Li}(d, p)\iso{12}{Li}$, $\iso{11}{Li}(d, d)\iso{11}{Li}$ e $\iso{11}{Li}(d, t)\iso{10}{Li}$. Usa a mesma enerxía do feixe para todas de \qty{7.5}{A\MeV}.
    
    A clase que vai facer iso é \lstinline|ActPhysics::Kinematics|. O criterio que debes usar para chamar ao constructor é o seguinte: \lstinline|("feixe", "target", "lixeira", "pesada", Tbeam)|, que configura a equivalencia coas fórmulas matemáticas do seguinte xeito: $$1 (\text{feixe}) + 2 (\text{target}) \longrightarrow 3 (\text{lixeira}) + 4 (\text{pesada})$$
    Ademais, \lstinline|GetKinematicLine3()| darache a representación $T_3$ vs $\theta_{Lab, 3}$ (polo tanto, da partícula marcada como \textit{lixeira}) e \lstinline|GetKinematicLine4()| o mesmo pero da 4, a pesada. Representa ambas no mesmo \textit{pad}.

    Vas ver como as formas son moi diferentes, o que forza a deseñar experimentos moi distintos en función da reacción a medir (ou un que sexa capaz de medir todas as canles de reacción).
    
    \item \textbf{Sampleado} de variables: Para a simulación vamos necesitar samplear variables gaussianas e uniformes. Crea un macro con 3 histogramas (que vas representar en 3 \textit{pads}) distintos nos que samplees as seguintes distribucións (podes xogar cos parámetros como queiras);
    \begin{itemize}
        \item Gaussiana: \lstinline|gRandom->Gaus(mean, sigma)|
        \item Uniforme: \lstinline|gRandom->Uniform(x0, x1)|
        \item O ángulo polar $\theta$ é especial se queres obter unha \href{https://mathworld.wolfram.com/SpherePointPicking.html}{distribución uniforme}:
        $$\theta = \arccos\left(\texttt{gRandom->Uniform(-1, 1)}\right)$$
    \end{itemize}
    Nota que este ángulo vai estar en radiáns. Para converter a graos, debes multiplicar polo factor de conversión correspondente. Se non o queres escribir, podes usar \lstinline|TMath::RadToDeg()| (a inversa é \lstinline|TMath::DegToRad()|) de ROOT.
\end{enumerate}

\end{document}
