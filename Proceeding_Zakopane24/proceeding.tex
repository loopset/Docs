\documentclass{appolb}
\usepackage{graphicx}
% graphicx package included for placing figures in the text
%------------------------------------------------------

\usepackage[version=4]{mhchem}
\newcommand{\iso}[2]{\ce{^{#1}#2}}
\usepackage[
style=phys,
articletitle=false,
biblabel=brackets,
%sorting=none,
maxbibnames=1,
]{biblatex}
\addbibresource{bibliography.bib}
%%%%%%%%%%%%%%%%%%%%%%%%%%%%%%%%%%%%%%%%%%%%%%%%%%
%                                                %
%    BEGINNING OF TEXT                           %
%                                                %
%%%%%%%%%%%%%%%%%%%%%%%%%%%%%%%%%%%%%%%%%%%%%%%%%%
\begin{document}
% \eqsec  % uncomment this line to get equations numbered by (sec.num)
\title{Quenching of spectroscopic factors in \iso{10,12}{Be}(d,\iso{3}{He}) reactions %
\thanks{Presented at the 57th Zakopane Conference on Nuclear Physics, {\it Extremes of the Nuclear Landscape}, Zakopane, Poland, 25 August–1 September, 2024.}%
% you can use '\\' to break lines
}
\author{M. Lozano-Gonz\'alez, B. Fern\'andez-Dom\'inguez, \\J. Lois-Fuentes
\address{IGFAE and USC}
\\[3mm]
{A. Matta, F. Delaunay % of different affiliation
\address{LPC-Caen}
}
}
\maketitle
\begin{abstract}
    Ola
\end{abstract}
  
\section{Introduction}
In the mean-field picture, nucleons inside nuclei move in single-particle orbitals with well-defined energies and quantum numbers. This approach, yet it describes many of the nuclear properties, fails to account for short-range correlations among nucleons. 

Transfer reactions, in which a single nucleon is added or removed from a core, provide a unique tool to probe the impact of those correlations in the single-particle strengths. For that regard, spectroscopic factors (SF) consitute a measure of the shell occupancies and thus they measure directly the magnitude of the SRC.


\section{Next section}
The text...
\subsection{Subsection}
The text...\cite{lois23}

%uncomment the following lines to place a figure
%\begin{figure}[htb]
%\centerline{%
%\includegraphics[width=12.5cm]{Fig1}}
%\caption{Plot of ...}
%\label{Fig:F2H}
%\end{figure}

\printbibliography
\end{document}

