\documentclass[11pt, a4paper, galician]{article}
\usepackage{babel}
\usepackage{fontspec}
\usepackage{libertinus}%
\usepackage{float}
\usepackage{parskip}
\usepackage[version=4]{mhchem}
\usepackage[figuresleft]{rotating}
\usepackage{graphicx}
\usepackage{amsmath}
\usepackage{slashed}
\usepackage{fancyhdr} 
\usepackage[dvipsnames, table]{xcolor}
\usepackage{wrapfig}
\usepackage[labelfont=bf, skip=4pt, figureposition=bottom, font=normalsize]{caption}
\usepackage{siunitx}
\usepackage{adjustbox}
\usepackage{booktabs} 
\usepackage{multirow}
\usepackage[a4paper,left=2cm,right=2cm,top=2.5cm,bottom=2.5cm]{geometry}
\usepackage{csquotes}
\usepackage{enumitem}
\usepackage{subcaption}
\usepackage{listings}
\usepackage{abstract}
\usepackage{hyperref} 
%%adapted to Galician (GZ)
%\renewcommand{\tableautorefname}{Cadro}
\usepackage{lipsum}

%% ENUMITEM settings
%\setlist{itemsep=4pt}
\setlist[enumerate,1]{
	label={\bfseries \arabic*.},
}

%% SIUNITX
%\sisetup{output-decimal-marker = {,}}
\sisetup{exponent-product= \cdot}
\sisetup{separate-uncertainty=false}
\sisetup{table-parse-only=true}
\sisetup{inter-unit-product = \ensuremath { { } \cdot { } } }
\sisetup{detect-all}
\sisetup{group-digits=integer}
\sisetup{print-unity-mantissa=false}
\sisetup{range-phrase = -}
\sisetup{range-units = single}
\sisetup{list-units = single}
\sisetup{propagate-math-font = true}
\sisetup{text-series-to-math = true}
\DeclareSIUnit{\espesor}{\milli\g\per\cm\squared}
\DeclareSIUnit{\aten}{\cm\squared\per\milli\g}
\DeclareSIUnit\clight{\text{\ensuremath{c}}}
\DeclareSIUnit\year{a}
\DeclareSIUnit{\barn}{b}
\DeclareSIUnit{\bar}{bar}

%%OTHER COMMANDS
%% for isotopes
\newcommand{\iso}[2]{\ce{^{#1}#2}}
%for vectors
\newcommand{\vect}[1]{\boldsymbol{#1}}
% figure sizes
\newcommand*{\fw}{0.6}
%%for the listing package
\definecolor{backcolour}{rgb}{0.95,0.95,0.92}
\lstset{
	%backgroundcolor=\color{backcolour},
	basicstyle=\normalsize\ttfamily,
	language=C++,
	keepspaces=false,
	keywordstyle=\bfseries\color{BurntOrange},
	morekeywords={VCluster, VFilter, TPCData, std, tuple, Point, Vector, 
    ActRoot, MergerData},
	stringstyle=\color{Orange},
	% identifierstyle=\color{ProcessBlue},
	breaklines=false,
	captionpos=b,
	% belowskip=-0.85 \baselineskip,
	%columns=flexible,
}

\title{\textbf{Clases de datos}}
\author{Miguel Lozano González\\
	}
\date{\empty}%\date{\today}

\pagestyle{fancy}
\fancyhf{}
%\fancyhead[R]{\thepage}
\fancyhead[R]{\textsc{actroot manual}}
\fancyfoot[C]{\thepage}
\renewcommand{\headrulewidth}{0.75pt}
%\renewcommand{\footrulewidth}{0.75pt}
\setlength{\headheight}{14.5pt}

\hypersetup{
	colorlinks=true,
	linkcolor=blue,
	filecolor=magenta,      
	urlcolor=cyan,
}

% %% BIBLIOGRAPHY
% \usepackage[
% backend=biber,
% style=numeric,
% sorting=none
% ]{biblatex}
% \addbibresource{./biblio.bib}

%% DOCUMENT

\begin{document}
\begin{minipage}{0.48\linewidth}
	\maketitle
\end{minipage}\hfill
\begin{minipage}{0.48\linewidth}
	\tableofcontents
\end{minipage}

\noindent\rule{\textwidth}{1pt}

\section{\textit{MergerData}}
Os datos finais están na clase \lstinline{MergerData} de \lstinline{ActRoot}, á cal non se pode acceder sen instalar o código. Porén, o \lstinline{TTree} segue sendo accesíbel. As ramas son:
\begin{itemize}
    \item \textit{fRun, fEntry}: run e entry asociados ao \lstinline|TTree| de Thomas.
    \item \textit{fThetaLight}: ángulo respecto do \textbf{beam} da partícula \textbf{lixeira}. En \textbf{graos}. Coa corrección de Juan.
    \item \textit{fPhiLight}: da partícula \textbf{lixeira}. En \textbf{graos}. Sendo $\vec{t}$ o vector dirección unitario da traza, calcúlase como:
    \begin{equation*}
        \phi = \arctan\left(\frac{t_y}{t_z}\right)
    \end{equation*}
    \textcolor{red}{\textbf{Non estou seguro de que estea ben calculado así}}.
    \item \textit{fSilLayers, fSilEs, fSilNs}: Información dos \textbf{silicios}: \textit{layer} (no caso de que haxa varias), enerxía e número. Para o experimento de Juan hai varias: \lstinline|l0| (á esquerda), \lstinline|f0| (a primeira frontal) e \lstinline|f1| (a segunda frontal). Son \textbf{vectores}: cada \textit{hit} no silicio ten un elemento \textit{layer}, enerxía e índice.
    \item \textit{fSP}: é o \textbf{punto de impacto no silicio} en \unit{\mm}. Podes acceder por separado ás coordenadas nas ramas \lstinline|fSP.fCoordinates.[fX, fY, fZ]|
    \item \textit{fRP}: é o \textbf{vértice de reacción} en \unit{\mm}. Mesmo comportamento que o anterior.
    \item \textit{fQave}: \textbf{Carga promedio} da traza (i.e., dividida pola lonxitude de traza) e \textbf{corrixida} pola dependencia en Z. En unidades arbitrarias \unit{\cdot\per\mm}.
    \item \textit{fTrackLength}: \textbf{Lonxitude \textcolor{red}{total} da traza}. En \unit{\mm}.
\end{itemize}

E máis cousas, pero iso é o básico. Algúns comentarios:
\begin{itemize}
    \item O código analiza todos os eventos, sexan válidos ou non. Polo tanto, unha grande parte son eventos nulos: as variables anteriores toman valores por defecto.
    \item Para recuperar os eventos físicos o mellor é facer unha gate nos silicios, considerando só aqueles nos que \lstinline|fSilLayers.size() > 0|.
    \item Se queres traballar coa capa \textit{f0}, podes gatear da seguinte forma:
    \begin{lstlisting}
        if(fSilLayers.size() > 0)// ou == 1
        {
            if(std::find(fSilLayers.begin(), fSilLayers.end(), "f0") != fSilLayers.end())
            {
                // o evento ten enerxía na capa 0
                // ...
            }
        }
    \end{lstlisting}
    \item Como o cálculo de $\phi$ non sei se está ben, sempre se pode recalcular, pois sábese a posición no vértice e no silicio.
    \item Sen ter o ficheiro \lstinline|#include "ActMergerData.h"| vai haber un error ao ler o \lstinline|TTree|, pero aínda así poderase facer 
    \begin{lstlisting}
        float var {};
        tree->SetBranchAddres("fRP.fCoordinates.fX", &var);// exemplo
    \end{lstlisting}
    \item Algunhas variables son gardadas como \lstinline|float| en lugar de \lstinline|double| para aforrar espazo; se hai erro, próbese co outro tipo.
\end{itemize}
\end{document}