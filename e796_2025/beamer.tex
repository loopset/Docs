\documentclass[aspectratio=43, dvipsnames]{beamer}
\usepackage[version=4]{mhchem}
\usepackage[aboveskip=1pt, belowskip=1pt]{caption}
\usepackage{amsmath}
\usepackage{siunitx}
\usepackage{emoji}
\usepackage{pgfplots}
\pgfplotsset{compat=1.18,}

% Command to display isotopes
\newcommand{\iso}[2]{\ce{^{#1}#2}}
% Command to write overlap
\newcommand{\overlap}[2]{\left\langle#1\middle\vert#2\right\rangle}
% Command to write Rs
\newcommand{\rs}{$\text{R}_{\text{S}}$ }
% Command to draw enumitems outside the environment
\newcommand{\enumitem}[1]{%
\setcounter{enumi}{#1}\usebeamertemplate{enumerate item}% 
}
\newcommand{\boxitem}[1]{\raisebox{0.15em}{\enumitem{#1}}}
% Set caption package options
\captionsetup{labelformat=empty}

\title[Oxygen spectroscopy]{Low-lying spectroscopy of 20O}
\date[Feb 2025]{February 2025}
\author[M. Lozano et al.]{M. Lozano-González, B. Fernández-Domínguez, \texorpdfstring{\newline}{} J. Lois-Fuentes, F. Delaunay}
\institute{USC-IGFAE and LPC-Caen}

\usetheme{igfae}

\begin{document}

\maketitle

\section{Motivation}
\begin{frame}{A recap on spectroscopic factors}
	\textbf{Spectroscopic factors} shed light on the occupancy of single-particle states:
	\begin{equation*}
		\left.\frac{d\sigma}{d\Omega}\right\vert_{\text{exp}} = C^{2}S \cdot \left.\frac{d\sigma}{d\Omega}\right\vert_{\text{s.p}}, \quad \sum C^{2}S = (2j + 1) \text{ in IPSM}
	\end{equation*}
	\begin{columns}[T]
		\begin{column}{0.48\linewidth}
			\hfill{}
			\begin{beamercolorbox}[sep=0.75em, center, wd=0.85\linewidth,rounded=true]{box1}
				\textbf{Experimentally:} Reduction of \sim\qty{65}{\percent}!
			\end{beamercolorbox}%
			\hfill{}
			\begin{itemize}
				\item \textbf{Short-range} correlations: tensor forces,...
				\item \textbf{Long-range}: vibrations, giant resonances,...
			\end{itemize}
		\end{column}
		\begin{column}{0.48\linewidth}
			\vspace{-1em}
			\begin{figure}
				\includegraphics[width=0.725\linewidth]{example-image}
				\caption{L. Lapikás, Nuclear Phys. A 553 (1993)}
			\end{figure}
		\end{column}
	\end{columns}
\end{frame}

\section{Experimental setup}
\begin{frame}{Experimental setup}
	E796 was performed at LISE (GANIL) back in March 2022 under these experimental conditions:
	\begin{columns}[t]
		\begin{column}{0.5\linewidth}
			\begin{itemize}
				\item Beam: \iso{20}{O} @ \qty{35}{A\MeV}
				\item Gas: \qty{90}{\percent}\ce{D_2} and \qty{10}{\percent} \ce{iC_4H_{10}}
				\item Silicons: two front layers and one left. \qty{500}{\micro\m}-thick
			\end{itemize}
			\mycolorbox[0.4]{box2}{
				\textbf{Neutron removal}\\
				$\iso{20}{O}(\iso{}{p}, \iso{}{d})$\\
				$\iso{20}{O}(\iso{}{d}, \iso{}{t})$
			}\hfill
			\mycolorbox[0.4]{box3}{
				\textbf{Proton removal}\\
				$\iso{20}{O}(\iso{}{d},\iso{3}{He})$
			}
		\end{column}\hfil
		\begin{column}{0.45\linewidth}
			\begin{figure}
				\begin{center}
					\includegraphics[width=0.95\textwidth]{figures/setup.pdf}
				\end{center}
			\end{figure}

		\end{column}
	\end{columns}

\end{frame}

\section{Results}
\begin{frame}{Results: (in)elastic scattering}
	These are the excitation energy spectra for protons and deuterons.
	\begin{figure}
		\includegraphics[width=0.95\textwidth]{figures/elastic_xz.pdf}
		% \caption{}
	\end{figure}
	\begin{columns}[c]
		\begin{column}{0.45\linewidth}
			\mycolorbox{box1}{Only 1st excited state}
		\end{column}%
		\begin{column}{0.45\linewidth}
			\mycolorbox{box2}{
				Up to 7 $E_{x} > 0$ states observed!
			}
		\end{column}
	\end{columns}
\end{frame}

\begin{frame}{Results: \iso{20}{O}(d,d)}
	Angular distributions for the \textbf{ground-state} and first excited states:
	\begin{figure}
		\includegraphics[width=0.85\linewidth]{figures/d_ang.pdf}
	\end{figure}
	\mycolorbox{box3}{Remaining states: low stats. Coming soon.}
\end{frame}

\begin{frame}{Results: \iso{20}{O}(p,p)}
	For the proton scattering:
	\begin{figure}
		\includegraphics[width=0.85\textwidth]{figures/p_ang.pdf}
	\end{figure}
	\begin{columns}[c]
		\begin{column}{0.55\linewidth}
			\mycolorbox{box2}{
				\textbf{Issue}: gs not reproduced by any OMP!\\
				\emoji{cold-sweat}
			}
		\end{column}%
		\begin{column}{0.45\linewidth}
			\mycolorbox{box4}{
				1st excited as well?
			}
		\end{column}
	\end{columns}

\end{frame}

\begin{frame}{About normalizations}
	Just to recall the xs formula:
	\begin{equation*}
		\frac{d\sigma}{d\Omega} = \frac{N}{ \textcolor{red}{N_{\text{beam}}N_{\text{targets}}} \epsilon \Delta \Omega} = \frac{N}{ \textcolor{red}{\alpha} \epsilon \Delta \Omega}
	\end{equation*}
	\begin{columns}[c]
		\begin{column}{0.5\linewidth}
			\mycolorbox[1]{box2}{
				\begin{itemize}
					\item $N_{\text{beam}} \leftarrow$ CFA counter
					\item $N_{\text{targets}} \leftarrow$ Gas mixture. Sensitive to p.
				\end{itemize}
				Theo. lines need \textbf{scaling} ($\alpha$) to match experimental data\\
				$\alpha$ in agreement with Juan's\\
				$\Rightarrow$ Not likely $\epsilon$ issue
				% 	% Not likely an issue of reconstruction $\epsilon$
			}
		\end{column}%
		\begin{column}{0.5\linewidth}
			\mycolorbox[1]{box4}{
				Which norm should we use?
				\begin{figure}
					\includegraphics[width=0.9\textwidth]{figures/theo.pdf}
				\end{figure}
				Protons are more "reliable" \emoji{thinking}
			}
		\end{column}
	\end{columns}
\end{frame}

\begin{frame}{Results: \iso{20}{O}(d,t)}
	Excited states are populated up to $\sim\qty{15}{\MeV}$:
	\begin{figure}
		\includegraphics[width=0.95\textwidth]{figures/dt_xs.pdf}
	\end{figure}
	\mycolorbox{box3}{
		1n and 2n \textbf{phase spaces} are included in the fit. Small (p,d) contamination as $\sim \qty{16}{\MeV}$ under control.
	}
\end{frame}

\begin{frame}{Results: \iso{20}{O}(d,t)}
	\only<+>{
		\begin{figure}
			\includegraphics[width=0.95\textwidth]{figures/ang_0.pdf}
		\end{figure}
		\mycolorbox{box1}{
			g.s fits well to $\Delta L = 2$.
		}
  }%
  \only<+>{
    \begin{columns}[c]
    \begin{column}{0.6\linewidth}
    \begin{figure}
      \includegraphics[height=0.825\textheight]{figures/ang_1.pdf}
    \end{figure}
    
    \end{column}%
    \begin{column}{0.4\linewidth}
    \mycolorbox{box3}{
      Few stats for some. Rebinning is foreseen.
    }\\ 
    \mycolorbox{box4}{
      Almost all are $\Delta L = 1$!
    }
    \end{column}
    \end{columns}
  }
\end{frame}
\end{document}
